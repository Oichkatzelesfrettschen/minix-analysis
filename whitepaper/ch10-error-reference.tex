% ===============================================================================
% CHAPTER 10: ERROR REFERENCE AND TROUBLESHOOTING GUIDE
% ===============================================================================

\chapter{Error Reference and Troubleshooting}
\label{ch:errorreference}

This chapter provides comprehensive reference for the 15-error detection framework, quick lookup guides, troubleshooting procedures, and recovery mechanisms.

The general error detection and recovery workflow is illustrated in Figure~\ref{fig:error-recovery-flowchart}, which shows how the system detects errors, classifies them, and executes appropriate recovery procedures.

\begin{figure}[!htbp]
\centering
\begin{tikzpicture}[scale=0.95]
    % Start
    \node[decision] (detect) at (4, 9) {Error\\Detected?};

    % Detection methods
    \node[action] (method1) at (1.5, 7.5) {Return Value\\Check};
    \node[action] (method2) at (4, 7.5) {State\\Assertion};
    \node[action] (method3) at (6.5, 7.5) {Memory\\Boundary};

    \draw[arrow] (detect) -- (method1);
    \draw[arrow] (detect) -- (method2);
    \draw[arrow] (detect) -- (method3);
    \draw[arrow] (method1) -- (4, 6.5);
    \draw[arrow] (method2) -- (4, 6.5);
    \draw[arrow] (method3) -- (4, 6.5);

    % Classification
    \node[decision] (classify) at (4, 6) {Classify Using\\15-Error Taxonomy};

    \draw[arrow] (detect) -- (classify);

    % Recovery path decision
    \node[decision] (recovery) at (4, 4.5) {Recovery\\Type?};

    \draw[arrow] (classify) -- (recovery);

    % Recovery types
    \node[active] (auto) at (0.5, 2.5) {Automatic\\Recovery};
    \node[action] (guided) at (2.5, 2.5) {Guided\\Recovery};
    \node[warning] (manual) at (4.5, 2.5) {Manual\\Recovery};
    \node[error] (critical) at (6.5, 2.5) {Critical\\Halt};

    \draw[arrow] (recovery) -- (auto);
    \draw[arrow] (recovery) -- (guided);
    \draw[arrow] (recovery) -- (manual);
    \draw[arrow] (recovery) -- (critical);

    % Recovery actions
    \node[action] (autoact) at (0.5, 1) {Retry\\Restart\\Reclaim};

    \node[action] (guideact) at (2.5, 1) {Diagnose\\Propose\\Confirm};

    \node[action] (manualact) at (4.5, 1) {Log State\\Notify Admn\\Wait};

    \node[error] (halt) at (6.5, 1) {System\\Halt};

    \draw[arrow] (auto) -- (autoact);
    \draw[arrow] (guided) -- (guideact);
    \draw[arrow] (manual) -- (manualact);
    \draw[arrow] (critical) -- (halt);

    % Recovery verification
    \node[decision] (verify) at (2.5, -0.5) {Recovery\\Success?};

    \draw[arrow] (autoact) -- (verify);
    \draw[arrow] (guideact) -- (verify);
    \draw[arrow] (manualact) -- (verify);

    % Success/Failure outcomes
    \node[active] (success) at (1, -2) {System\\Continues};

    \node[error] (failure) at (4, -2) {Escalate\\to Critical};

    \draw[arrow] (verify) -- node[anchor=south] {Yes} (success);
    \draw[arrow] (verify) -- node[anchor=north] {No} (failure);

    % Logging
    \node[data] (log) at (1, -3.5) {Log Event};
    \draw[arrow] (success) -- (log);
    \draw[arrow] (failure) -- (log);

\end{tikzpicture}
\caption{Error Detection and Recovery Flowchart. When an error is detected, the system classifies it using the 15-error taxonomy, selects an appropriate recovery strategy (automatic, guided, manual, or critical halt), executes recovery actions, verifies success, and logs the event. Escalation occurs if recovery fails.}
\label{fig:error-recovery-flowchart}
\end{figure}

% ===============================================================================
\section{Error Classification Framework}
% ===============================================================================

\subsection{15-Error Taxonomy}

The error detection framework classifies system errors into 15 categories:

\begin{enumerate}
\item \textbf{Boot-Time Initialization Errors}
  \begin{itemize}
  \item Detection during kernel startup phases
  \item Causes: Hardware missing, firmware incompatible, memory errors
  \item Recovery: Diagnostic output, alternate boot paths
  \end{itemize}

\item \textbf{Memory Management Errors}
  \begin{itemize}
  \item Allocation failures, page table corruption, segmentation
  \item Causes: Insufficient memory, fragmentation, protection violations
  \item Recovery: Memory reclamation, service restart
  \end{itemize}

\item \textbf{Inter-Process Communication (IPC) Errors}
  \begin{itemize}
  \item Message delivery failure, protocol violations, queue overflow
  \item Causes: Dead service, message buffer full, invalid destination
  \item Recovery: Service restart, message queue flush
  \end{itemize}

\item \textbf{Interrupt and Exception Handling Errors}
  \begin{itemize}
  \item Invalid interrupt vectors, unhandled exceptions
  \item Causes: Hardware fault, invalid IDT, privilege violations
  \item Recovery: Interrupt masking, handler reload
  \end{itemize}

\item \textbf{System Call Parameter Validation Errors}
  \begin{itemize}
  \item Invalid parameters, out-of-range values, malformed requests
  \item Causes: Application bug, malicious input, protocol mismatch
  \item Recovery: Parameter rejection with EINVAL, audit log
  \end{itemize}

\item \textbf{Driver and Service Process Errors}
  \begin{itemize}
  \item Device driver failures, service init errors
  \item Causes: Hardware not ready, missing firmware, startup failure
  \item Recovery: Driver restart, fallback to alternate
  \end{itemize}

\item \textbf{Privilege Level Violation Errors}
  \begin{itemize}
  \item Unauthorized kernel access, ring violations
  \item Causes: Application bug, security violation, transition error
  \item Recovery: Process termination, audit log
  \end{itemize}

\item \textbf{Context Switch and Register Corruption Errors}
  \begin{itemize}
  \item Corrupted process state, invalid register values
  \item Causes: Stack overflow, buffer overrun, memory write error
  \item Recovery: Process abort, register restoration
  \end{itemize}

\item \textbf{State Machine Invalid Transition Errors}
  \begin{itemize}
  \item Kernel state machine enters invalid state
  \item Causes: Race condition, protocol violation, state corruption
  \item Recovery: State reset to known-good value
  \end{itemize}

\item \textbf{Resource Exhaustion Errors}
  \begin{itemize}
  \item Process table full, file descriptor limits, memory limits
  \item Causes: Too many processes, resource leak, DOS condition
  \item Recovery: Process cleanup, resource adjustment
  \end{itemize}

\item \textbf{Configuration Inconsistency Errors}
  \begin{itemize}
  \item Invalid kernel configuration, incompatible settings
  \item Causes: Manual config editing, version mismatch, corruption
  \item Recovery: Configuration validation, defaults restoration
  \end{itemize}

\item \textbf{Data Structure Corruption Errors}
  \begin{itemize}
  \item Corrupted kernel structures (process table, file table)
  \item Causes: Memory corruption, out-of-bounds write, hardware error
  \item Recovery: Structure rebuild from backup, system halt
  \end{itemize}

\item \textbf{Timeout and Deadlock Condition Errors}
  \begin{itemize}
  \item System calls hanging, circular wait conditions
  \item Causes: Deadlock between processes, unresponsive hardware
  \item Recovery: Timeout interrupt, process force termination
  \end{itemize}

\item \textbf{Recovery Mechanism Failure Errors}
  \begin{itemize}
  \item Recovery procedure itself fails
  \item Causes: Recovery dependencies missing, cascading failures
  \item Recovery: Escalate to manual intervention
  \end{itemize}

\item \textbf{Synchronization and Race Condition Errors}
  \begin{itemize}
  \item Improper mutual exclusion, race conditions
  \item Causes: Missing locks, lock ordering violations, timing windows
  \item Recovery: Retry with synchronization, abort operation
  \end{itemize}
\end{enumerate}

% ===============================================================================
\section{Quick Reference Guide}
% ===============================================================================

\subsection{Symptom-Based Error Lookup}

\begin{description}
\item[System won't boot] See Boot-Time Initialization Errors (Error \#1)
\item[Out of memory] See Memory Management Errors (Error \#2)
\item[Service crashes] See IPC Errors (\#3) or Service Errors (\#6)
\item[Permission denied] See Privilege Violation Errors (Error \#7)
\item[Segmentation fault] See Memory Errors (\#2) or Context Switch Errors (\#8)
\item[System call fails] See Parameter Validation Errors (Error \#5)
\item[System hangs] See Timeout Errors (\#13) or Race Conditions (\#15)
\item[System unstable] See Data Corruption (\#12) or State Machine Errors (\#9)
\end{description}

\subsection{Error Severity Levels}

Errors classified by severity:

\begin{description}
\item[CRITICAL] System must halt immediately (data corruption, privilege violation)
\item[HIGH] Service or process must restart (IPC failure, resource exhaustion)
\item[MEDIUM] Operation fails but system continues (parameter error, timeout)
\item[LOW] Warning condition, operation retries (transient failure, retriable)
\end{description}

% ===============================================================================
\section{Troubleshooting Procedures}
% ===============================================================================

\subsection{Boot Failure Troubleshooting}

If system fails to boot:

\begin{enumerate}
\item Check console output for error messages
\item Note exact error message and location in boot sequence
\item Refer to Error Taxonomy above to identify error type
\item Follow recovery procedure for that error
\item If recovery fails, gather system state for analysis
\end{enumerate}

\TODO{Provide actual boot failure logs and diagnosis examples}

\subsection{Runtime Error Troubleshooting}

If error occurs during normal operation:

\begin{enumerate}
\item Identify affected service or process
\item Check system logs (kernel ring buffer or system log file)
\item Determine if automatic recovery occurred
\item If automatic recovery failed, perform manual recovery
\item Report issue if problem persists
\end{enumerate}

\TODO{Document log file locations and analysis techniques}

\subsection{Performance Troubleshooting}

If system performs poorly:

\begin{enumerate}
\item Check CPU utilization and load
\item Monitor memory usage (free memory, swap)
\item Verify disk I/O performance
\item Check IPC message queue depths
\item Identify CPU hotspots via profiling
\end{enumerate}

\TODO{Provide performance diagnostic tools}

% ===============================================================================
\section{Recovery Procedures}
% ===============================================================================

\subsection{Automatic Recovery}

Many errors trigger automatic recovery:

\begin{enumerate}
\item \textbf{Detection:} Error condition detected by kernel
\item \textbf{Classification:} Error mapped to recovery category
\item \textbf{Action:} Recovery procedure executed:
  \begin{itemize}
  \item Service restart
  \item Memory reclamation
  \item Queue flush
  \item State reset
  \end{itemize}
\item \textbf{Verification:} Confirm recovery succeeded
\item \textbf{Logging:} Record action for diagnostics
\end{enumerate}

\TODO{Document all automatic recovery procedures}

\subsection{Guided Recovery}

Some errors require user confirmation:

\begin{enumerate}
\item System detects error and proposes recovery
\item User is informed of proposed action
\item User confirms or rejects recovery
\item If accepted: Execute recovery procedure
\item If rejected: Log decision and escalate
\end{enumerate}

\TODO{Provide recovery proposal examples}

\subsection{Manual Recovery}

Critical errors require manual intervention:

\begin{enumerate}
\item System halts to prevent damage
\item Error state written to console and log
\item Administrator reviews error and state
\item Administrator executes manual recovery:
  \begin{itemize}
  \item Service restart from command line
  \item Configuration correction
  \item System reset
  \end{itemize}
\end{enumerate}

\TODO{Create manual recovery procedures}

% ===============================================================================
\section{Chapter Summary}
% ===============================================================================

This chapter provided comprehensive reference material for error detection, troubleshooting, and recovery. The 15-error taxonomy, quick lookup guides, and recovery procedures enable rapid error identification and appropriate remediation.

\chapter{Boot Variant: bsp\_finish\_booting()}

\section{Overview}

The \texttt{bsp\_finish\_booting()} function (line 38 of main.c) initializes the Bootstrap Processor
(BSP) before multi-processor support. On single-processor systems, this is the only initialization path.
On multi-processor systems, this function runs on the BSP while other cores are waiting.

\section{WHAT: BSP Initialization Steps}

\begin{enumerate}
\item \textbf{Clear FPU}: Reset floating-point unit state
\item \textbf{Enable FPU Features}: SSE, AVX if available
\item \textbf{Set Up CPU-Local Data}: Per-CPU variables and structures
\item \textbf{Initialize Lapic (Local APIC)}: If multi-processor system
\item \textbf{Set Up Performance Counters}: If supported by CPU
\end{enumerate}

\section{HOW: Source Code Analysis}

\begin{lstlisting}[style=cstyle,caption={bsp\_finish\_booting() from main.c:38}]
void bsp_finish_booting(void)
{
  /* Bootstrap processor specific initialization */

  /* Clear FPU state and enable features */
  arch_fpu_init();

  /* Set up per-CPU variables */
  setup_cpu_local_vars();

  /* Initialize LAPIC for multi-processor support */
#ifdef USE_APIC
  lapic_init();
#endif

  /* CPU feature detection and reporting */
  cpu_print_freq();
}
\end{lstlisting}

\section{Timing}

Typical duration: 1-3 milliseconds
\begin{itemize}
\item FPU init: 0.5-1 ms
\item CPU-local setup: 0.2-0.5 ms
\item APIC init: 0.5-1.5 ms (if applicable)
\item Feature detection: 0.3-0.5 ms
\end{itemize}

\section{Summary}

\texttt{bsp\_finish\_booting()} prepares the Bootstrap Processor for operation, including
FPU initialization, per-CPU data setup, and multi-processor infrastructure.

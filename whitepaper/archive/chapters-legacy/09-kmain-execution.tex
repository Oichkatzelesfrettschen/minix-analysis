\chapter{Kernel Orchestration: kmain() Execution}

\section{Overview}

Chapter 3 provided a high-level overview of kmain() orchestration. This chapter
focuses on the detailed execution path, process table setup, and transition to the scheduler.

\section{Process Table Initialization}

The process table (proc array in main.c) is initialized with:

\begin{enumerate}
\item NR\_TASKS kernel tasks (interrupt handlers, etc.)
\item NR\_SYS\_PROCS system processes (filesystem, network, etc.)
\item NR\_USER\_PROCS user processes (later added by VM)
\end{enumerate}

Typical configuration:
\begin{verbatim}
NR_TASKS:      10-15 (kernel tasks)
NR_SYS_PROCS:  5-8 (system servers)
NR_USER_PROCS: 128 (user processes)
Total:         150-160 process slots
\end{verbatim}

\section{Boot Process Setup}

For each boot process:

\begin{enumerate}
\item Assign process ID and endpoint
\item Extract multiboot module info (start address, size)
\item Initialize privilege structure
\item Set scheduling flags (RTS\_PROC\_STOP, RTS\_NO\_PRIV, etc.)
\item Call architecture-specific setup (arch\_boot\_proc)
\end{enumerate}

\section{Scheduling Flags}

Process states during boot:

\begin{table}[h!]
\centering
\caption{Process Scheduling Flags at Boot}
\begin{tabular}{ll}
\toprule
Flag & Meaning \\
\midrule
RTS\_PROC\_STOP & Process inhibited (waiting for scheduler) \\
RTS\_NO\_PRIV & No privileges assigned yet (waiting for root process) \\
RTS\_NO\_QUANTUM & No CPU time quantum (scheduler inhibited) \\
RTS\_VMINHIBIT & Inhibited until VM sets up page table \\
RTS\_BOOTINHIBIT & Boot-time inhibition \\
\bottomrule
\end{tabular}
\end{table}

Only kernel tasks and the root system process are immediately schedulable.
All other processes remain inhibited until the root process grants privileges.

\section{Summary}

kmain() creates the foundation for multitasking by:
\begin{enumerate}
\item Initializing all process structures
\item Loading boot processes from multiboot modules
\item Assigning privileges and capabilities
\item Preparing for scheduler activation
\end{enumerate}

\documentclass[12pt]{article}
\usepackage[margin=1in,landscape]{geometry}
\usepackage{tikz}
\usepackage{xcolor}
\usepackage{standalone}
\usepackage{graphicx}
\usepackage{pdfpages}

\usetikzlibrary{shapes.geometric,arrows.meta,positioning,fit,patterns,calc,matrix}

\title{MINIX 3 CPU Interface Architecture\\Visual Reference}
\author{Comprehensive Analysis of Microkernel-CPU Interaction}
\date{2025-10-30}

\begin{document}

\maketitle

\newpage
\section{System Call Flow}
This diagram shows the complete path of a system call from user space through the CPU hardware
transitions, kernel processing, scheduling, and return to user space (possibly a different process).

\vspace{1cm}
\includepdf[pages=-,fitpaper]{01-system-call-flow.pdf}

\newpage
\section{Context Switch Architecture}
This diagram illustrates the three phases of a context switch: the state before switching (Process A),
the kernel's switching operation including the critical CR3 modification that changes address spaces,
and the state after switching (Process B).

\vspace{1cm}
\includepdf[pages=-,fitpaper]{02-context-switch.pdf}

\newpage
\section{Privilege Ring Architecture}
This diagram shows MINIX 3's use of x86 protection rings, highlighting that only Ring 0 (kernel)
and Ring 3 (user/servers/drivers) are utilized. It illustrates the various gates between privilege
levels and the IPC message flow that goes through the kernel.

\vspace{1cm}
\includepdf[pages=-,fitpaper]{03-privilege-rings.pdf}

\newpage
\section{Summary}

These diagrams visualize the key CPU interface mechanisms in MINIX 3:

\begin{itemize}
    \item \textbf{System Call Flow}: Shows all three entry mechanisms (INT, SYSENTER, SYSCALL) and the
complete path through the kernel including context saving, C handlers, scheduling, and context restoration.

    \item \textbf{Context Switch}: Details the register and memory state changes that occur when the kernel
switches between processes, with emphasis on the CR3 write that causes a TLB flush and address space change.

    \item \textbf{Privilege Rings}: Illustrates the protection model with Ring 0 (microkernel) and Ring 3
(everything else), showing the gates that allow controlled transitions between privilege levels.
\end{itemize}

\subsection{Key Files Referenced}

\begin{description}
    \item[\texttt{mpx.S}] (652 lines) - All entry/exit points, interrupt/exception handlers, context restoration
    \item[\texttt{klib.S}] (798 lines) - Privileged instructions, CR3 manipulation, context switching
    \item[\texttt{protect.c}] (361 lines) - GDT/IDT/TSS initialization and management
    \item[\texttt{proc.c}] (1900+ lines) - Process management, scheduling, IPC
    \item[\texttt{exception.c}] (240 lines) - Exception handling and CR2 access
\end{description}

\subsection{Critical CPU Operations}

\begin{itemize}
    \item \textbf{LGDT/LIDT/LTR}: Initialize CPU descriptor tables (once at boot, or per-CPU for SMP)
    \item \textbf{MOV to/from CR3}: Switch page directories (every context switch between different address spaces)
    \item \textbf{INVLPG}: Selective TLB invalidation (faster than full CR3 reload)
    \item \textbf{IRET/SYSEXIT/SYSRET}: Return from kernel to user space with privilege level change
    \item \textbf{CLI/STI}: Disable/enable interrupts for critical sections
\end{itemize}

\subsection{Compilation}

These diagrams were created with TikZ/PGFPlots for publication-quality rendering. To regenerate:

\begin{verbatim}
cd /home/eirikr/Playground/minix-cpu-analysis/diagrams
pdflatex 01-system-call-flow.tex
pdflatex 02-context-switch.tex
pdflatex 03-privilege-rings.tex
pdflatex master-diagrams.tex
\end{verbatim}

\end{document}

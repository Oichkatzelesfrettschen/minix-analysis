\documentclass[tikz,border=10pt]{standalone}

\usepackage{tikz}
\usetikzlibrary{shapes,arrows,positioning,calc,backgrounds,fit,chains,decorations.pathreplacing}

% MINIX TikZ style guide color palette
\definecolor{primaryblue}{RGB}{41,128,185}
\definecolor{secondarygreen}{RGB}{39,174,96}
\definecolor{accentorange}{RGB}{230,126,34}
\definecolor{warningred}{RGB}{192,57,43}
\definecolor{lightgray}{RGB}{236,240,241}
\definecolor{darkgray}{RGB}{52,73,94}

\begin{document}
\begin{tikzpicture}[
    % Node styles
    step/.style={rectangle, draw=accentorange, fill=accentorange!15, thick, minimum width=7cm, minimum height=0.8cm, font=\small, align=left},
    substep/.style={rectangle, draw=primaryblue, fill=primaryblue!10, thick, minimum width=6.5cm, minimum height=0.7cm, font=\footnotesize, align=left},
    data/.style={rectangle, draw=secondarygreen, fill=secondarygreen!10, thick, minimum width=4cm, minimum height=0.6cm, font=\footnotesize\ttfamily, align=left},
    decision/.style={diamond, draw=warningred, fill=warningred!10, thick, minimum width=2cm, minimum height=1.5cm, font=\footnotesize, align=center, aspect=2},
    label/.style={font=\small\bfseries, align=left},
    arrow/.style={->, >=stealth, thick},
    note/.style={font=\footnotesize\itshape, align=left, text=darkgray},
    codebox/.style={rectangle, draw=darkgray, fill=lightgray!50, thick, font=\tiny\ttfamily, align=left},
]

% Title
\node[font=\LARGE\bfseries] at (10, 23) {kmain() Initialization Detailed Flow};
\node[font=\large] at (10, 22.2) {MINIX Kernel Bootstrap Process (main.c:115-328)};

% Source reference
\node[font=\footnotesize] at (10, 21.5) {\textbf{Source:} minix/kernel/main.c};

% ============================================================================
% LEFT COLUMN: Main kmain() flow
% ============================================================================

% Entry point
\node[step, fill=warningred!20] (entry) at (5, 20.5) {
\textbf{void kmain(kinfo\_t *local\_cbi)} \hfill \textit{main.c:115}\\
Entry from bootloader (EBX = multiboot info)
};

% BSS sanity check
\node[step] (bss_check) at (5, 19.5) {
1. BSS Sanity Check \hfill \textit{main.c:123-125}\\
assert(bss\_test == 0); bss\_test = 1;
};

\draw[arrow] (entry) -- (bss_check);

% Copy boot parameters
\node[step] (copy_params) at (5, 18.5) {
2. Copy Boot Parameters \hfill \textit{main.c:128-129}\\
memcpy(\&kinfo, local\_cbi, sizeof(kinfo));
};

\draw[arrow] (bss_check) -- (copy_params);

% cstart()
\node[step] (cstart) at (5, 17.5) {
3. \textbf{cstart()} - Low-level Init \hfill \textit{main.c:147}
};

\draw[arrow] (copy_params) -- (cstart);

% BKL_LOCK()
\node[step] (bkl) at (5, 16.5) {
4. BKL\_LOCK() \hfill \textit{main.c:149}\\
Acquire Big Kernel Lock (SMP safety)
};

\draw[arrow] (cstart) -- (bkl);

% proc_init()
\node[step] (proc_init) at (5, 15.5) {
5. \textbf{proc\_init()} \hfill \textit{main.c:157}\\
Clear process table, set up mappings
};

\draw[arrow] (bkl) -- (proc_init);

% IPCF_POOL_INIT()
\node[step] (ipcf) at (5, 14.5) {
6. IPCF\_POOL\_INIT() \hfill \textit{main.c:158}\\
Initialize IPC filter pool
};

\draw[arrow] (proc_init) -- (ipcf);

% Boot modules check
\node[decision] (boot_check) at (5, 13.2) {Module\\count\\match?};

\draw[arrow] (ipcf) -- (boot_check);

\node[step, fill=warningred!20, minimum width=3cm] (panic) at (1, 13.2) {
\textbf{panic()}\\
Boot error
};

\draw[arrow] (boot_check) -- node[above, font=\tiny] {No} (panic);

% Boot image loop
\node[step] (boot_loop) at (5, 11.8) {
7. Boot Image Loop \hfill \textit{main.c:165-272}\\
for (i=0; i < NR\_BOOT\_PROCS; ++i)
};

\draw[arrow] (boot_check) -- node[right, font=\tiny] {Yes} (boot_loop);

% Set up process
\node[substep] (setup_proc) at (5, 10.8) {
• Get process attributes: ip = \&image[i]\\
• Get process pointer: rp = proc\_addr(ip->proc\_nr)\\
• Copy name (tasks only), set endpoint
};

\draw[arrow] (boot_loop) -- (setup_proc);

% Privilege assignment
\node[decision] (sched_check) at (5, 9.3) {Kernel\\task or\\VM?};

\draw[arrow] (setup_proc) -- (sched_check);

% Set privileges
\node[substep] (set_priv) at (5, 7.9) {
\textbf{get\_priv(rp, static\_priv\_id(proc\_nr))}\\
• Set s\_flags (TSK\_F, VM\_F, RSYS\_F)\\
• Set s\_trap\_mask (allowed traps)\\
• fill\_sendto\_mask (IPC targets)\\
• Set kernel call mask
};

\draw[arrow] (sched_check) -- node[right, font=\tiny] {Yes} (set_priv);

% No privilege
\node[substep, fill=warningred!10] (no_priv) at (2, 7.9) {
RTS\_SET(rp,\\
RTS\_NO\_PRIV |\\
RTS\_NO\_QUANTUM)
};

\draw[arrow] (sched_check) -- node[above, font=\tiny] {No} (no_priv);

% Arch boot
\node[substep] (arch_boot) at (5, 6.7) {
\textbf{arch\_boot\_proc(ip, rp)} \hfill \textit{main.c:257}\\
Architecture-specific process setup
};

\draw[arrow] (set_priv) -- (arch_boot);
\draw[arrow] (no_priv) |- (arch_boot);

% Set RTS flags
\node[substep] (rts_flags) at (5, 5.7) {
• rp->p\_rts\_flags |= RTS\_VMINHIBIT (if not VM)\\
• rp->p\_rts\_flags |= RTS\_BOOTINHIBIT\\
• rp->p\_rts\_flags |= RTS\_PROC\_STOP\\
• rp->p\_rts\_flags \&= \textasciitilde RTS\_SLOT\_FREE
};

\draw[arrow] (arch_boot) -- (rts_flags);

% End loop
\node[step] (end_loop) at (5, 4.5) {
End loop (all boot processes initialized)
};

\draw[arrow] (rts_flags) -- (end_loop);

% arch_post_init()
\node[step] (arch_post) at (5, 3.5) {
8. \textbf{arch\_post\_init()} \hfill \textit{main.c:283}\\
Architecture-specific post-initialization
};

\draw[arrow] (end_loop) -- (arch_post);

% Register IPC names
\node[step] (ipc_names) at (5, 2.5) {
9. Register IPC Call Names \hfill \textit{main.c:285-290}\\
IPCNAME(SEND), IPCNAME(RECEIVE), etc.
};

\draw[arrow] (arch_post) -- (ipc_names);

% memory_init() and system_init()
\node[step] (sys_init) at (5, 1.5) {
10. \textbf{memory\_init() + system\_init()} \hfill \textit{main.c:293-296}\\
Initialize memory subsystem and system processes
};

\draw[arrow] (ipc_names) -- (sys_init);

% add_memmap()
\node[step] (add_mem) at (5, 0.5) {
11. \textbf{add\_memmap()} \hfill \textit{main.c:301}\\
Free bootstrap memory to allocator
};

\draw[arrow] (sys_init) -- (add_mem);

% SMP check
\node[decision] (smp_check) at (5, -0.8) {CONFIG\\\_SMP?};

\draw[arrow] (add_mem) -- (smp_check);

% SMP init
\node[substep] (smp_init) at (3, -2.2) {
\textbf{smp\_init()}\\
Initialize all CPUs
};

\draw[arrow] (smp_check) -- node[left, font=\tiny] {Yes} (smp_init);

% bsp_finish_booting()
\node[step, fill=secondarygreen!20] (bsp_finish) at (5, -3.5) {
12. \textbf{bsp\_finish\_booting()} \hfill \textit{main.c:316/324}\\
Final boot steps, switch to Ring 3
};

\draw[arrow] (smp_check) -- node[right, font=\tiny] {No} (bsp_finish);
\draw[arrow] (smp_init) |- (bsp_finish);

% NOT_REACHABLE
\node[step, fill=warningred!20] (not_reach) at (5, -4.5) {
\textbf{NOT\_REACHABLE;} \hfill \textit{main.c:327}\\
kmain() never returns
};

\draw[arrow] (bsp_finish) -- (not_reach);

% ============================================================================
% RIGHT COLUMN: cstart() detail
% ============================================================================

\node[label] at (15, 20.5) {\textbf{cstart() Details} (main.c:403-475)};

\node[codebox, minimum width=8cm] (cstart_code) at (15, 18.5) {
// Low-level initialization\\
void cstart(void) \{\\
\ \ /* Segment setup */\\
\ \ prot\_init(); \ \ \ \ \ \ \ \ \ \ \ \  // GDT, IDT, TSS\\
\\
\ \ /* Boot verbosity */\\
\ \ verboseboot = atoi(env\_get(VERBOSEBOOTVARNAME));\\
\\
\ \ /* Clock initialization */\\
\ \ init\_clock();\\
\\
\ \ /* Get user stack/data limits */\\
\ \ kinfo.user\_sp = USR\_STACKTOP\_COMPACT;\\
\ \ kinfo.user\_end = USR\_DATATOP\_COMPACT;\\
\\
\ \ /* Record system info */\\
\ \ kinfo.nr\_procs = NR\_PROCS;\\
\ \ kinfo.nr\_tasks = NR\_TASKS;\\
\ \ strlcpy(kinfo.release, OS\_RELEASE, ...);\\
\\
\ \ /* Check APIC/SMP config */\\
\ \ config\_no\_apic = atoi(env\_get("no\_apic"));\\
\ \ config\_no\_smp = atoi(env\_get("no\_smp"));\\
\\
\ \ /* Initialize interrupt handling */\\
\ \ intr\_init(0);\\
\\
\ \ /* Architecture-specific init */\\
\ \ arch\_init();\\
\}
};

% ============================================================================
% RIGHT COLUMN: bsp_finish_booting() detail
% ============================================================================

\node[label] at (15, 3.5) {\textbf{bsp\_finish\_booting()} (main.c:38-109)};

\node[codebox, minimum width=8cm] (bsp_code) at (15, 0) {
void bsp\_finish\_booting(void) \{\\
\ \ /* Identify CPU features */\\
\ \ cpu\_identify(); \ \ \ \ \ \ \ \ \ \ \ \ \ // CPUID detection\\
\\
\ \ /* VM not running yet */\\
\ \ vm\_running = 0;\\
\\
\ \ /* Setup random number generator */\\
\ \ krandom.random\_sources = RANDOM\_SOURCES;\\
\\
\ \ /* Set current process pointer to idle */\\
\ \ proc\_ptr = \&idle\_proc;\\
\ \ bill\_ptr = \&idle\_proc;\\
\\
\ \ /* Print startup banner */\\
\ \ announce(); \ \ \ \ \ \ \ \ \ \ \ \ \ \ \ \ \ // "MINIX 3.4.0 ..."\\
\\
\ \ /* Unblock boot processes */\\
\ \ for (i=0; i < NR\_BOOT\_PROCS - NR\_TASKS; i++)\\
\ \ \ \ RTS\_UNSET(proc\_addr(i), RTS\_PROC\_STOP);\\
\\
\ \ /* Initialize CPU accounting */\\
\ \ cycles\_accounting\_init();\\
\\
\ \ /* Start timer interrupts */\\
\ \ boot\_cpu\_init\_timer(system\_hz);\\
\\
\ \ /* Initialize FPU */\\
\ \ fpu\_init();\\
\\
\ \ /* Set SMP flags */\\
\ \ cpu\_set\_flag(bsp\_cpu\_id, CPU\_IS\_READY);\\
\\
\ \ /* Kernel can no longer allocate */\\
\ \ kernel\_may\_alloc = 0;\\
\\
\ \ /* \textbf{CRITICAL: Switch to Ring 3} */\\
\ \ switch\_to\_user(); \ \ \ \ \ \ \ \ \ \ \ // Never returns\\
\ \ NOT\_REACHABLE;\\
\}
};

% ============================================================================
% Data Structures (Top Right)
% ============================================================================

\node[label] at (15, 13) {\textbf{Key Data Structures}};

\node[data] (kinfo_struct) at (15, 12.2) {
kinfo\_t kinfo\\
• boot params\\
• memory map\\
• boot modules
};

\node[data] (proc_table) at (15, 11.3) {
proc[] table\\
• NR\_PROCS entries\\
• Process metadata
};

\node[data] (boot_image) at (15, 10.4) {
boot\_image[]\\
• Kernel tasks\\
• System servers\\
• Drivers
};

\node[data] (priv_table) at (15, 9.5) {
priv[] table\\
• Privilege structs\\
• IPC permissions
};

% ============================================================================
% Process States (Bottom Right)
% ============================================================================

\node[label] at (15, -2) {\textbf{Process RTS Flags}};

\node[note, text width=7cm] at (15, -3.2) {
RTS\_SLOT\_FREE - slot unused\\
RTS\_PROC\_STOP - process stopped\\
RTS\_NO\_PRIV - no privileges assigned\\
RTS\_NO\_QUANTUM - no CPU time\\
RTS\_VMINHIBIT - wait for VM pagetable\\
RTS\_BOOTINHIBIT - wait for boot completion
};

% Background zones
\begin{pgfonlayer}{background}
    % Main kmain flow
    \node[fill=accentorange!5, rounded corners, fit=(entry) (not_reach)] {};

    % cstart detail
    \node[fill=primaryblue!5, rounded corners, fit=(cstart_code)] {};

    % bsp_finish_booting detail
    \node[fill=secondarygreen!5, rounded corners, fit=(bsp_code)] {};

    % Data structures
    \node[fill=lightgray, rounded corners, fit=(kinfo_struct) (priv_table)] {};
\end{pgfonlayer}

% Timeline marker
\draw[decorate, decoration={brace, amplitude=8pt}]
    (0.5, 20.5) -- (0.5, -4.5) node[midway, left=10pt, align=center, font=\footnotesize\bfseries, rotate=90] {Boot Time: 50-200 ms};

\end{tikzpicture}
\end{document}

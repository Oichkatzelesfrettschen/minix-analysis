\documentclass[tikz,border=10pt]{standalone}
\usepackage{tikz}
\usetikzlibrary{shapes,arrows,positioning,calc,decorations.pathmorphing,backgrounds,shadows,fit}

% Define colors
\definecolor{primaryblue}{RGB}{0,102,204}
\definecolor{secondarygreen}{RGB}{46,204,113}
\definecolor{accentorange}{RGB}{255,127,0}
\definecolor{warningred}{RGB}{231,76,60}
\definecolor{lightgray}{RGB}{236,240,241}
\definecolor{darkgray}{RGB}{52,73,94}

\begin{document}
\begin{tikzpicture}[
    box/.style={rectangle, rounded corners=2pt, draw=primaryblue, fill=primaryblue!15, minimum width=5.5cm, minimum height=0.7cm, font=\small, align=center},
    hw/.style={rectangle, rounded corners=2pt, draw=warningred, fill=warningred!20, minimum width=5.5cm, minimum height=0.7cm, font=\small, align=center},
    kernelbox/.style={rectangle, rounded corners=2pt, draw=secondarygreen, fill=secondarygreen!15, minimum width=5.5cm, minimum height=0.7cm, font=\small, align=center},
    procbox/.style={rectangle, draw=darkgray, fill=yellow!15, minimum width=3cm, minimum height=2cm, font=\footnotesize\ttfamily, align=left},
    regbox/.style={rectangle, draw=accentorange, fill=accentorange!10, minimum width=2.5cm, minimum height=0.5cm, font=\footnotesize\ttfamily},
    arrow/.style={->, >=stealth, thick},
    flow/.style={arrow, primaryblue},
    hwflow/.style={arrow, warningred},
    dataflow/.style={<->, >=stealth, thick, accentorange, line width=1.2pt},
    label/.style={font=\footnotesize}
]

% Title
\node[font=\Large\bfseries] at (8, 18) {MINIX Context Switch: Process A → Process B};
\node[font=\normalsize] at (8, 17.5) {Complete CPU state transition with CR3 (page directory) switch};

% Current state: Process A
\node[font=\bfseries] at (3, 16.5) {Process A (Current)};
\node[regbox] (reg_a_eip) at (3, 15.8) {EIP: 0x0804abcd};
\node[regbox] (reg_a_esp) at (3, 15.2) {ESP: 0xbffff800};
\node[regbox] (reg_a_cr3) at (3, 14.6) {CR3: 0x01000000};
\node[regbox] (reg_a_eax) at (3, 14.0) {EAX: 42};
\node[label] at (3, 13.5) {+ all other GPRs};

% Kernel entry (already in kernel from syscall/interrupt)
\node[kernelbox] (kern1) at (8, 16) {Kernel: In syscall handler};
\node[label] at (11.5, 16) {(Process A already saved)};

% Scheduler decision
\node[kernelbox] (kern2) at (8, 15.2) {Kernel: switch\_to\_user()};
\node[kernelbox] (kern3) at (8, 14.4) {Kernel: pick\_proc()};
\node[label, text width=6cm, align=left] at (12, 14.4) {
    Scan run queues by priority\\
    Select next runnable process
};

% Process table lookup
\node[kernelbox] (kern4) at (8, 13.5) {Returns: proc\_table[B] pointer};

% Context switch function
\node[kernelbox, fill=accentorange!20, draw=accentorange] (kern5) at (8, 12.6) {arch\_finish\_switch\_to\_user(\&proc\_B)};
\node[font=\footnotesize\bfseries, accentorange] at (8, 12.1) {Critical CPU manipulation begins};

% Process table structures
\node[procbox] (proc_a) at (3, 10.5) {
proc\_table[A]:\\
\ \ EIP: 0x0804abcd\\
\ \ ESP: 0xbffff800\\
\ \ CR3: 0x01000000\\
\ \ EAX: 42\\
\ \ ... (all regs)\\
\ \ trap\_style: KTS\_INT
};

\node[procbox] (proc_b) at (13, 10.5) {
proc\_table[B]:\\
\ \ EIP: 0x0805ffee\\
\ \ ESP: 0xbfffe000\\
\ \ CR3: 0x02000000\\
\ \ EAX: 1337\\
\ \ ... (all regs)\\
\ \ trap\_style: KTS\_SYSENTER
};

% CR3 switch (THE KEY OPERATION)
\node[hw, fill=warningred!30, minimum width=11cm] (cr3_1) at (8, 8.5) {
ASM: movl P\_CR3(\%edx), \%eax \ \ \ // Load proc\_B's page dir addr
};
\node[hw, fill=warningred!30, minimum width=11cm] (cr3_2) at (8, 7.8) {
ASM: mov \%cr3, \%ecx \ \ \ \ \ \ \ \ \ \ \ \ \ \ // Read current CR3
};
\node[hw, fill=warningred!30, minimum width=11cm] (cr3_3) at (8, 7.1) {
ASM: cmp \%eax, \%ecx \ \ \ \ \ \ \ \ \ \ \ \ \ \ // Same address space?
};
\node[hw, fill=warningred!30, minimum width=11cm] (cr3_4) at (8, 6.4) {
ASM: je 4f \ \ \ \ \ \ \ \ \ \ \ \ \ \ \ \ \ \ \ \ \ \ \ \ // Skip if same
};
\node[hw, fill=warningred!30, minimum width=11cm, line width=2pt] (cr3_5) at (8, 5.7) {
ASM: mov \%eax, \%cr3 \ \ \ \ \ \ \ \ \ \ \ \ \ // ★ SWITCH PAGE TABLES ★
};

% Hardware effect
\node[hw, fill=yellow!30, minimum width=11cm] (tlb_flush) at (8, 4.8) {
CPU Hardware: TLB flush (all non-global entries invalidated)
};
\node[label, warningred, font=\bfseries] at (3, 4.8) {~100 cycles};

% TSS update
\node[kernelbox] (tss_update) at (8, 3.9) {
Update TSS.ESP0 ← proc\_B.kernel\_stack\_top
};
\node[label] at (12.5, 3.9) {For next interrupt};

% Restore segments
\node[kernelbox] (seg1) at (8, 3.1) {Load segment registers from proc\_B};
\node[regbox] at (12, 3.1) {DS, ES, FS, GS};

% Check trap style
\node[kernelbox, fill=primaryblue!20] (trap_check) at (8, 2.2) {
Check proc\_B.trap\_style
};

% Three return paths
\node[kernelbox, minimum width=3cm] (ret_int) at (3, 1.2) {KTS\_INT → IRET};
\node[kernelbox, minimum width=3cm] (ret_sysenter) at (8, 1.2) {KTS\_SYSENTER → SYSEXIT};
\node[kernelbox, minimum width=3cm] (ret_syscall) at (13, 1.2) {KTS\_SYSCALL → SYSRET};

% Restore registers
\node[kernelbox] (restore1) at (8, 0.3) {RESTORE\_GP\_REGS from proc\_B};
\node[kernelbox] (restore2) at (8, -0.4) {Pop/load all GPRs: EAX-EDI, EBP};

% Return to user
\node[hw] (final_ret) at (8, -1.2) {CPU: Return instruction (IRET/SYSEXIT/SYSRET)};
\node[hw] (cpu_restore) at (8, -2.0) {CPU: Restore EIP, ESP, EFLAGS, CS, SS};

% Process B resumes
\node[font=\bfseries] at (13, -2.9) {Process B (New)};
\node[regbox] (reg_b_eip) at (13, -3.6) {EIP: 0x0805ffee};
\node[regbox] (reg_b_esp) at (13, -4.2) {ESP: 0xbfffe000};
\node[regbox] (reg_b_cr3) at (13, -4.8) {CR3: 0x02000000};
\node[regbox] (reg_b_eax) at (13, -5.4) {EAX: 1337};
\node[box, minimum width=4cm] (proc_b_running) at (13, -6.2) {Process B resumes execution};

% Flow arrows
\draw[flow] (kern1) -- (kern2);
\draw[flow] (kern2) -- (kern3);
\draw[flow] (kern3) -- (kern4);
\draw[flow] (kern4) -- (kern5);
\draw[flow] (kern5) -- (cr3_1);
\draw[hwflow] (cr3_1) -- (cr3_2);
\draw[hwflow] (cr3_2) -- (cr3_3);
\draw[hwflow] (cr3_3) -- (cr3_4);
\draw[hwflow, line width=2pt] (cr3_4) -- (cr3_5);
\draw[hwflow, line width=2pt] (cr3_5) -- (tlb_flush);
\draw[flow] (tlb_flush) -- (tss_update);
\draw[flow] (tss_update) -- (seg1);
\draw[flow] (seg1) -- (trap_check);
\draw[flow] (trap_check) -- (ret_int);
\draw[flow] (trap_check) -- (ret_sysenter);
\draw[flow] (trap_check) -- (ret_syscall);
\draw[flow] (ret_int) -- (restore1);
\draw[flow] (ret_sysenter) -- (restore1);
\draw[flow] (ret_syscall) -- (restore1);
\draw[flow] (restore1) -- (restore2);
\draw[flow] (restore2) -- (final_ret);
\draw[hwflow] (final_ret) -- (cpu_restore);
\draw[hwflow] (cpu_restore) -- (proc_b_running);

% Data flows
\draw[dataflow] (proc_a) -- node[above, label] {saved} (3, 12.6);
\draw[dataflow] (proc_b) -- node[above, label] {loaded} (13, 12.6);
\draw[dataflow, bend left=20] (proc_b.west) to node[above, label, sloped] {CR3 value} (cr3_1.east);

% Background highlighting
\begin{scope}[on background layer]
    \fill[lightgray!30] (0, 18.5) rectangle (16, -6.8);
    % Highlight critical CR3 switch
    \fill[warningred!10] (2, 9) rectangle (14, 4.5);
    \node[font=\footnotesize\bfseries, warningred] at (1.5, 7) {Critical Section};
\end{scope}

% Side annotations
\node[font=\footnotesize, rotate=90, secondarygreen] at (0.3, 14) {Scheduler};
\node[font=\footnotesize, rotate=90, warningred] at (0.3, 7) {CPU State Switch};
\node[font=\footnotesize, rotate=90, primaryblue] at (0.3, 0) {Context Restore};

% Key points box
\node[rectangle, draw=accentorange, fill=accentorange!10, text width=14.5cm, align=left, font=\footnotesize] at (8, -7.5) {
\textbf{Critical CPU Operations:}\\
1. **CR3 Write**: Most expensive operation (~100 cycles) + TLB flush invalidates all cached page table entries\\
2. **TLB Flush**: Necessary evil - next process would see wrong virtual→physical mappings otherwise\\
3. **TSS.ESP0 Update**: Must point to new process's kernel stack for next interrupt\\
4. **Segment Register Reload**: User segments must match new process's GDT/LDT entries\\
5. **No Lazy Context Switch**: MINIX saves/restores ALL state (Linux lazy FPU, MINIX doesn't)\\
6. **Three Return Paths**: IRET (legacy), SYSEXIT (Intel), SYSRET (AMD) - chosen by saved trap\_style
};

% File references
\node[font=\footnotesize, text=darkgray, align=left] at (4, -8.8) {
\texttt{klib.S:586-651} arch\_finish\_switch\_to\_user\\
\texttt{klib.S:609-621} CR3 conditional switch\\
\texttt{proc.c} switch\_to\_user(), pick\_proc()
};

\node[font=\footnotesize, text=darkgray, align=left] at (12, -8.8) {
\texttt{mpx.S:434} restore\_user\_context\_int\\
\texttt{mpx.S:391} restore\_user\_context\_sysenter\\
\texttt{mpx.S:414} restore\_user\_context\_syscall
};

\end{tikzpicture}
\end{document}

\documentclass[tikz,border=10pt]{standalone}

\usepackage{tikz}
\usetikzlibrary{shapes,arrows,positioning,calc,backgrounds,fit,decorations.pathreplacing}

% MINIX TikZ style guide color palette
\definecolor{primaryblue}{RGB}{41,128,185}
\definecolor{secondarygreen}{RGB}{39,174,96}
\definecolor{accentorange}{RGB}{230,126,34}
\definecolor{warningred}{RGB}{192,57,43}
\definecolor{lightgray}{RGB}{236,240,241}
\definecolor{darkgray}{RGB}{52,73,94}

\begin{document}
\begin{tikzpicture}[
    % Node styles
    step/.style={rectangle, draw=accentorange, fill=accentorange!15, thick, minimum width=6cm, minimum height=0.8cm, font=\small, align=left},
    regbox/.style={rectangle, draw=secondarygreen, fill=secondarygreen!10, thick, minimum width=4.5cm, minimum height=0.6cm, font=\footnotesize\ttfamily, align=left},
    stackbox/.style={rectangle, draw=primaryblue, fill=primaryblue!10, thick, minimum width=3cm, minimum height=0.5cm, font=\footnotesize, align=center},
    codeblock/.style={rectangle, draw=darkgray, fill=lightgray!50, thick, font=\tiny\ttfamily, align=left, minimum width=5.5cm},
    label/.style={font=\small\bfseries, align=center},
    arrow/.style={->, >=stealth, thick},
    widearrow/.style={->, >=stealth, very thick, line width=1.5pt},
    note/.style={font=\footnotesize\itshape, align=left, text=darkgray},
]

% Title
\node[font=\LARGE\bfseries] at (10, 21) {Kernel to Userspace Transition};
\node[font=\large] at (10, 20.2) {Ring 0 → Ring 3 Privilege Level Change (switch\_to\_user)};

% Source reference
\node[font=\footnotesize] at (10, 19.5) {\textbf{Source:} arch-specific code (i386: arch/i386/mpx.S, switch\_to\_user)};

% ============================================================================
% LEFT COLUMN: Kernel State (Ring 0)
% ============================================================================

\node[label] at (5, 18.5) {\textbf{Kernel State (Ring 0)}};

% Current CPU state
\node[step] (kernel_state) at (5, 17.5) {
\textbf{Before switch\_to\_user()}\\
CPL = 0 (Ring 0 - kernel mode)
};

% Segment registers (kernel)
\node[label, font=\footnotesize] at (5, 16.5) {Segment Registers};

\node[regbox] (cs_kernel) at (5, 15.9) {CS = KERNEL\_CS (0x08)\\DPL = 0, RPL = 0};
\node[regbox] (ds_kernel) at (5, 15.1) {DS = KERNEL\_DS (0x10)\\DPL = 0, RPL = 0};
\node[regbox] (ss_kernel) at (5, 14.3) {SS = KERNEL\_DS (0x10)\\Stack in kernel space};

% Process to switch to
\node[step] (select_proc) at (5, 13.2) {
Select first ready process:\\
proc\_ptr = get\_next\_ready\_process()
};

\draw[arrow] (ss_kernel) -- (select_proc);

% Setup user context
\node[step] (setup_context) at (5, 12.1) {
Prepare user process context:\\
• Load p\_reg (saved registers)\\
• Get user stack pointer\\
• Get user code entry point
};

\draw[arrow] (select_proc) -- (setup_context);

% Build user stack frame
\node[step] (build_frame) at (5, 10.8) {
Build stack frame for IRET:\\
Push to kernel stack:
};

\draw[arrow] (setup_context) -- (build_frame);

% Stack frame components
\node[stackbox] (stack_ss) at (5, 10) {SS (User stack segment)};
\node[stackbox] (stack_esp) at (5, 9.4) {ESP (User stack pointer)};
\node[stackbox] (stack_eflags) at (5, 8.8) {EFLAGS (with IF=1)};
\node[stackbox] (stack_cs) at (5, 8.2) {CS (User code segment)};
\node[stackbox] (stack_eip) at (5, 7.6) {EIP (User entry point)};

\draw[arrow] (build_frame) -- (stack_ss);
\draw[arrow] (stack_ss) -- (stack_esp);
\draw[arrow] (stack_esp) -- (stack_eflags);
\draw[arrow] (stack_eflags) -- (stack_cs);
\draw[arrow] (stack_cs) -- (stack_eip);

% Load data segments
\node[step] (load_segments) at (5, 6.7) {
Load user data segments:\\
mov ax, USER\_DS\_SELECTOR\\
mov ds, ax; mov es, ax
};

\draw[arrow] (stack_eip) -- (load_segments);

% IRET instruction
\node[step, fill=warningred!20] (iret) at (5, 5.5) {
\textbf{IRET} (Interrupt Return)\\
Pop CS, EIP, EFLAGS, SS, ESP\\
\textbf{CPL changes: 0 → 3}
};

\draw[widearrow, warningred] (load_segments) -- (iret);

% ============================================================================
% RIGHT COLUMN: User State (Ring 3)
% ============================================================================

\node[label] at (15, 18.5) {\textbf{User State (Ring 3)}};

% Current CPU state
\node[step, fill=secondarygreen!20] (user_state) at (15, 17.5) {
\textbf{After IRET}\\
CPL = 3 (Ring 3 - user mode)
};

\draw[widearrow, secondarygreen] (iret) -- ++(4,0) |- (user_state)
    node[pos=0.5, above, font=\tiny] {Privilege transition};

% Segment registers (user)
\node[label, font=\footnotesize] at (15, 16.5) {Segment Registers};

\node[regbox] (cs_user) at (15, 15.9) {CS = USER\_CS (0x1B)\\DPL = 3, RPL = 3};
\node[regbox] (ds_user) at (15, 15.1) {DS = USER\_DS (0x23)\\DPL = 3, RPL = 3};
\node[regbox] (ss_user) at (15, 14.3) {SS = USER\_DS (0x23)\\Stack in user space};

\draw[arrow] (user_state) -- (cs_user);

% Instruction pointer
\node[step] (user_eip) at (15, 13.2) {
EIP = process entry point\\
(usually \_start in user program)
};

\draw[arrow] (ss_user) -- (user_eip);

% Process begins execution
\node[step, fill=primaryblue!20] (user_exec) at (15, 12.1) {
\textbf{Process begins execution}\\
First instruction in user code
};

\draw[arrow] (user_eip) -- (user_exec);

% System call mechanism
\node[step] (syscall_entry) at (15, 10.8) {
When process needs kernel:\\
• INT 0x33 (legacy)\\
• SYSENTER (Intel)\\
• SYSCALL (AMD)
};

\draw[arrow] (user_exec) -- (syscall_entry);

% ============================================================================
% CENTER: Code examples
% ============================================================================

\node[label] at (10, 4.5) {\textbf{Assembly Code: switch\_to\_user()}};

\node[codeblock] (asm_code) at (10, 2.5) {
; arch/i386/mpx.S - switch\_to\_user\\
\\
switch\_to\_user:\\
\ \ ; Get current process pointer\\
\ \ mov eax, [proc\_ptr]\\
\\
\ \ ; Load user segment selectors\\
\ \ mov cx, USER\_DS\_SELECTOR | 3  ; RPL=3\\
\ \ mov ds, cx\\
\ \ mov es, cx\\
\ \ mov fs, cx\\
\ \ mov gs, cx\\
\\
\ \ ; Build IRET stack frame\\
\ \ push USER\_DS\_SELECTOR | 3    ; SS\\
\ \ push [eax + P\_STACKTOP]      ; ESP\\
\ \ pushf                          ; EFLAGS\\
\ \ push USER\_CS\_SELECTOR | 3    ; CS\\
\ \ push [eax + P\_PC]             ; EIP\\
\\
\ \ ; Atomic privilege transition\\
\ \ iret                           ; Ring 0 -> Ring 3\\
\\
\ \ ; \textbf{Never returns - now in Ring 3}
};

% ============================================================================
% BOTTOM: GDT Entry details
% ============================================================================

\node[label] at (5, 0.5) {\textbf{GDT Entries}};

\node[codeblock, minimum width=7cm] (gdt_entries) at (5, -1.3) {
Kernel Code (0x08):  Base=0, Limit=4GB, DPL=0, Type=Code\\
Kernel Data (0x10):  Base=0, Limit=4GB, DPL=0, Type=Data\\
User Code   (0x18):  Base=0, Limit=4GB, DPL=3, Type=Code\\
\ \ \ \ \ \ \ \ \ \ \ \ \ \ \ \ \ \ \ (Selector = 0x18 | 3 = 0x1B)\\
User Data   (0x20):  Base=0, Limit=4GB, DPL=3, Type=Data\\
\ \ \ \ \ \ \ \ \ \ \ \ \ \ \ \ \ \ \ (Selector = 0x20 | 3 = 0x23)\\
\\
RPL (Requestor Privilege Level) = bottom 2 bits\\
DPL (Descriptor Privilege Level) = segment descriptor\\
CPL (Current Privilege Level) = CS.RPL
};

\node[label] at (15, 0.5) {\textbf{Privilege Checks}};

\node[note, text width=7cm] (priv_checks) at (15, -1.3) {
IRET privilege check:\\
1. New CS.DPL must be $>=$ current CPL\\
2. New SS.DPL must equal new CS.DPL\\
3. New SS.RPL must equal new CS.RPL\\
4. Stack switch occurs if CPL changes\\
5. Interrupts re-enabled if EFLAGS.IF=1\\
\\
After IRET to Ring 3:\\
• Cannot execute privileged instructions\\
• Cannot access I/O ports (unless IOPL=3)\\
• Cannot modify CR0, CR3, etc.\\
• Must use syscall to re-enter kernel
};

% Process memory space note
\node[step, minimum width=8cm] at (10, -3.5) {
\textbf{Note:} Each process has its own CR3 (page directory)\\
loaded before switch\_to\_user(). All processes share kernel\\
code mapped at high addresses (kernel space).
};

% Background zones
\begin{pgfonlayer}{background}
    % Kernel side
    \node[fill=accentorange!5, rounded corners, fit=(kernel_state) (iret)] {};

    % User side
    \node[fill=secondarygreen!5, rounded corners, fit=(user_state) (syscall_entry)] {};

    % Code example
    \node[fill=lightgray, rounded corners, fit=(asm_code)] {};
\end{pgfonlayer}

% Ring transition annotation
\draw[decorate, decoration={brace, amplitude=10pt, mirror}]
    (1, 17.5) -- (1, 5.5) node[midway, left=12pt, align=center, font=\small\bfseries] {Ring 0};

\draw[decorate, decoration={brace, amplitude=10pt}]
    (19, 17.5) -- (19, 10.8) node[midway, right=12pt, align=center, font=\small\bfseries] {Ring 3};

% Critical instruction marker
\draw[warningred, very thick, dashed] (3, 5.9) rectangle (7, 5.1);
\node[font=\footnotesize\bfseries, text=warningred] at (5, 6.3) {Critical: CPL 0 $\rightarrow$ 3};

\end{tikzpicture}
\end{document}

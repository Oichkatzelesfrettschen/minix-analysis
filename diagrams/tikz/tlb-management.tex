\documentclass[tikz,border=10pt]{standalone}
\usepackage{tikz}
\usetikzlibrary{shapes,arrows,positioning,calc,decorations.pathmorphing,backgrounds,shadows,fit}

% Define colors
\definecolor{primaryblue}{RGB}{0,102,204}
\definecolor{secondarygreen}{RGB}{46,204,113}
\definecolor{accentorange}{RGB}{255,127,0}
\definecolor{warningred}{RGB}{231,76,60}
\definecolor{lightgray}{RGB}{236,240,241}
\definecolor{darkgray}{RGB}{52,73,94}

\begin{document}
\begin{tikzpicture}[
    box/.style={rectangle, rounded corners=2pt, draw=primaryblue, fill=primaryblue!15, minimum width=5cm, minimum height=0.7cm, font=\small, align=center},
    hw/.style={rectangle, rounded corners=2pt, draw=warningred, fill=warningred!20, minimum width=5cm, minimum height=0.7cm, font=\small, align=center},
    tlbentry/.style={rectangle, draw=darkgray, fill=yellow!15, minimum width=3cm, minimum height=0.5cm, font=\footnotesize\ttfamily, align=left},
    operation/.style={rectangle, rounded corners=3pt, draw=secondarygreen, fill=secondarygreen!15, minimum width=5cm, minimum height=0.8cm, font=\small\bfseries, align=center},
    arrow/.style={->, >=stealth, thick},
    label/.style={font=\footnotesize}
]

% Title
\node[font=\Large\bfseries] at (8, 17) {MINIX TLB Management Strategies};
\node[font=\normalsize] at (8, 16.5) {Translation Lookaside Buffer: Selective INVLPG vs Full CR3 Flush};

%% What is the TLB?
\node[font=\bfseries\large, primaryblue] at (8, 15.5) {TLB (Translation Lookaside Buffer)};

\node[rectangle, draw=primaryblue, fill=primaryblue!5, text width=14cm, align=left, font=\footnotesize] at (8, 14.3) {
\textbf{Purpose:} Cache for virtual→physical address translations to avoid slow page table walks\\
\textbf{Location:} Inside CPU, separate from L1/L2 cache\\
\textbf{Size:} Typically 64-512 entries (architecture-dependent)\\
\textbf{Structure:} Fully associative cache mapping {virtual page \#} → {physical page frame \# + flags}\\
\textbf{Problem:} When page tables change, TLB entries become stale and MUST be invalidated
};

% TLB entry example
\node[font=\bfseries] at (3, 12.8) {TLB Entry Example};
\node[tlbentry] (tlb1) at (3, 12) {VPN: 0x08048 (virtual)};
\node[tlbentry] (tlb2) at (3, 11.4) {PFN: 0x12345 (physical)};
\node[tlbentry] (tlb3) at (3, 10.8) {Flags: R/W, User, Present};
\node[tlbentry] (tlb4) at (3, 10.2) {Global: 0 (process-specific)};

\node[font=\bfseries] at (13, 12.8) {TLB After Page Unmap};
\node[tlbentry, fill=warningred!20] (tlb1_bad) at (13, 12) {VPN: 0x08048};
\node[tlbentry, fill=warningred!20] (tlb2_bad) at (13, 11.4) {PFN: 0x12345 ← STALE!};
\node[tlbentry, fill=warningred!20] (tlb3_bad) at (13, 10.8) {Flags: R/W... ← INVALID};
\node[label, warningred, font=\bfseries] at (13, 9.9) {Must invalidate!};

%% TWO STRATEGIES
\node[font=\Large\bfseries, accentorange] at (8, 9) {Two Invalidation Strategies};

%% LEFT: INVLPG (Selective)
\node[font=\bfseries\large, secondarygreen] at (4, 8) {INVLPG (Selective Flush)};

\node[operation] (invlpg_op) at (4, 7) {i386\_invlpg(vaddr)};

\node[box] (invlpg1) at (4, 6.1) {Kernel: Unmap single page};
\node[hw] (invlpg2) at (4, 5.3) {ASM: invlpg (\%eax)};
\node[label, text width=4.5cm, align=left] at (7.3, 5.3) {
    klib.S:549\\
    Invalidate one entry
};

\node[hw] (invlpg3) at (4, 4.4) {CPU: Search TLB for vaddr};
\node[hw] (invlpg4) at (4, 3.6) {CPU: Invalidate matching entry};
\node[hw] (invlpg5) at (4, 2.8) {CPU: Leave other entries cached};

\node[box, fill=secondarygreen!20] (invlpg_result) at (4, 1.9) {Result: 1 TLB entry flushed};

% Cost
\node[label, secondarygreen, font=\bfseries] at (4, 1.2) {Cost: ~1-3 cycles};
\node[label] at (4, 0.7) {Fast for single pages};

% Use cases
\node[rectangle, draw=secondarygreen, fill=secondarygreen!10, text width=6.5cm, align=left, font=\footnotesize] at (4, -0.5) {
\textbf{Use Cases:}\\
• Unmapping single page (mprotect)\\
• Page table entry modification\\
• Lazy allocation (map on demand)\\
• COW page replacement\\
\\
\textbf{Example:} VM server unmaps one page\\
→ Kernel calls i386\_invlpg(addr)\\
→ Only that page's TLB entry cleared
};

% Arrows
\draw[arrow] (invlpg_op) -- (invlpg1);
\draw[arrow] (invlpg1) -- (invlpg2);
\draw[arrow, warningred] (invlpg2) -- (invlpg3);
\draw[arrow, warningred] (invlpg3) -- (invlpg4);
\draw[arrow, warningred] (invlpg4) -- (invlpg5);
\draw[arrow] (invlpg5) -- (invlpg_result);

%% RIGHT: CR3 Reload (Full Flush)
\node[font=\bfseries\large, warningred] at (12, 8) {CR3 Reload (Full Flush)};

\node[operation, fill=warningred!20, draw=warningred] (cr3_op) at (12, 7) {mov \%eax, \%cr3};

\node[box] (cr3_1) at (12, 6.1) {Kernel: Context switch};
\node[hw] (cr3_2) at (12, 5.3) {ASM: mov P\_CR3(\%edx), \%eax};
\node[hw] (cr3_3) at (12, 4.5) {ASM: mov \%eax, \%cr3};
\node[label, text width=4.5cm, align=left] at (8.5, 4.5) {
    klib.S:618-621\\
    Load new page directory
};

\node[hw, fill=warningred!30] (cr3_4) at (12, 3.6) {CPU: Flush ALL non-global TLB entries};
\node[label, text width=4.5cm, align=right] at (15.5, 3.6) {
    Except entries with\\
    Global bit set
};
\node[hw] (cr3_5) at (12, 2.7) {CPU: Load new page directory base};
\node[hw] (cr3_6) at (12, 1.9) {CPU: Next translation = miss};

\node[box, fill=warningred!20] (cr3_result) at (12, 1.0) {Result: Entire TLB cleared};

% Cost
\node[label, warningred, font=\bfseries] at (12, 0.3) {Cost: ~100 cycles + refill penalty};
\node[label] at (12, -0.2) {Expensive but necessary};

% Use cases
\node[rectangle, draw=warningred, fill=warningred!10, text width=6.5cm, align=left, font=\footnotesize] at (12, -1.3) {
\textbf{Use Cases:}\\
• Context switch (different address space)\\
• Must flush all old process's mappings\\
• No way to selectively flush by PID\\
• Alternative: reload CR3 with same value\\
\ \ (forces full flush without changing PD)\\
\\
\textbf{Example:} Switch Process A → B\\
→ Load proc\_B.CR3 into CR3 register\\
→ ALL of Process A's TLB entries cleared\\
→ Process B starts with cold TLB
};

% Arrows
\draw[arrow] (cr3_op) -- (cr3_1);
\draw[arrow] (cr3_1) -- (cr3_2);
\draw[arrow] (cr3_2) -- (cr3_3);
\draw[arrow, warningred, line width=1.5pt] (cr3_3) -- (cr3_4);
\draw[arrow, warningred] (cr3_4) -- (cr3_5);
\draw[arrow, warningred] (cr3_5) -- (cr3_6);
\draw[arrow] (cr3_6) -- (cr3_result);

% Background
\begin{scope}[on background layer]
    \fill[lightgray!20] (0, 17.5) rectangle (16, -2.5);
\end{scope}

%% Performance Comparison
\node[font=\bfseries\large, accentorange] at (8, -3) {Performance Impact};

\node[rectangle, draw=accentorange, fill=accentorange!10, text width=15cm, align=left, font=\footnotesize] at (8, -4.8) {
\textbf{TLB Miss Penalty:} If TLB entry not found, CPU must do page table walk:

1. Read CR3 → get page directory base\\
2. Index with top 10 bits of vaddr → read page directory entry (PDE) — \textit{1 memory access}\\
3. Index with middle 10 bits → read page table entry (PTE) — \textit{1 memory access}\\
4. Combine PTE's physical frame with offset → final physical address\\
\textbf{Total: 2 extra memory accesses per TLB miss} (on x86 2-level paging)\\

\textbf{Post-Context-Switch Penalty:}\\
• Immediately after context switch, TLB is empty (all entries flushed)\\
• Every memory access initially misses → triggers page walk\\
• Typical workload: 64-512 misses before TLB warms up\\
• Cost: 64 misses × (2 memory accesses) × (50 cycles/access) ≈ 6400 cycles overhead\\
\textbf{This is why context switching is expensive in microkernels!}
};

% Optimization note
\node[rectangle, draw=primaryblue, fill=primaryblue!10, text width=15cm, align=left, font=\footnotesize] at (8, -7) {
\textbf{Optimization: Global Pages}\\
Some page table entries have a "Global" bit set (e.g., kernel code/data shared across all processes).\\
Global pages are NOT flushed on CR3 write, reducing post-switch TLB misses for kernel mappings.\\
MINIX uses global pages for kernel text/data (if CPU supports PGE - Page Global Enable in CR4).
};

% File references
\node[font=\footnotesize, text=darkgray, align=left] at (4, -8.2) {
\texttt{klib.S:549} i386\_invlpg\\
\texttt{arch\_do\_vmctl.c:56} VMCTL\_I386\_INVLPG\\
\texttt{mpx.S:594-595} reload\_cr3 (force flush)
};

\node[font=\footnotesize, text=darkgray, align=left] at (12, -8.2) {
\texttt{klib.S:618-621} CR3 conditional switch\\
\texttt{memory.c} Page table management\\
\texttt{archconst.h} I386\_VM\_PRESENT, PG\_G
};

\end{tikzpicture}
\end{document}

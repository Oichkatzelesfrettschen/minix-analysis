\documentclass[tikz,border=10pt]{standalone}

\usepackage{tikz}
\usetikzlibrary{shapes,arrows,positioning,calc,backgrounds,fit,decorations.pathreplacing,patterns}

% MINIX TikZ style guide color palette
\definecolor{primaryblue}{RGB}{41,128,185}
\definecolor{secondarygreen}{RGB}{39,174,96}
\definecolor{accentorange}{RGB}{230,126,34}
\definecolor{warningred}{RGB}{192,57,43}
\definecolor{lightgray}{RGB}{236,240,241}
\definecolor{darkgray}{RGB}{52,73,94}

\begin{document}
\begin{tikzpicture}[
    % Node styles
    cpu/.style={rectangle, draw=primaryblue, fill=primaryblue!20, thick, minimum width=3.5cm, minimum height=0.8cm, font=\small\bfseries, align=center},
    unit/.style={rectangle, draw=accentorange, fill=accentorange!15, thick, minimum width=3cm, minimum height=0.7cm, font=\footnotesize, align=center},
    cache/.style={rectangle, draw=secondarygreen, fill=secondarygreen!15, thick, minimum width=2.5cm, minimum height=0.6cm, font=\footnotesize, align=center},
    regbox/.style={rectangle, draw=darkgray, fill=lightgray, thick, minimum width=2cm, minimum height=0.5cm, font=\tiny\ttfamily, align=center},
    pipeline/.style={rectangle, draw=warningred, fill=warningred!10, thick, minimum width=4cm, minimum height=0.6cm, font=\footnotesize, align=center},
    instr/.style={rectangle, draw=primaryblue, fill=primaryblue!10, thick, minimum width=2.5cm, minimum height=0.5cm, font=\tiny, align=center},
    arrow/.style={->, >=stealth, thick},
    widearrow/.style={->, >=stealth, very thick, line width=1.5pt},
    doublearrow/.style={<->, >=stealth, thick},
    label/.style={font=\small\bfseries, align=center},
    note/.style={font=\footnotesize\itshape, align=left, text=darkgray},
]

% Title
\node[font=\LARGE\bfseries] at (10, 24) {Intel Pentium (P5) Microarchitecture};
\node[font=\large] at (10, 23.2) {First Superscalar x86 - Dual-Issue Pipeline (1993)};

% Specifications box
\node[note, text width=18cm, font=\footnotesize] at (10, 22.4) {
\textbf{Specifications:} 60-200 MHz, 3.1-3.3M transistors (0.8-0.35 \textmu m), 273-pin PGA,
8 KB L1 I-cache + 8 KB L1 D-cache, 64-bit data bus, 32-bit address bus (4 GB physical),
Superscalar (2-way), Branch prediction, APIC, FPU integrated, RDTSC instruction
};

% ============================================================================
% TOP: Pipeline Overview
% ============================================================================

\node[label] at (5, 21) {\textbf{Dual-Issue Superscalar Pipeline}};

% U-pipe (primary)
\node[pipeline] (pf_u) at (3, 20) {U-Pipe\\(Primary)};
\node[pipeline] (d1_u) at (3, 19.2) {D1: Decode 1};
\node[pipeline] (d2_u) at (3, 18.4) {D2: Decode 2};
\node[pipeline] (ex_u) at (3, 17.6) {EX: Execute};
\node[pipeline] (wb_u) at (3, 16.8) {WB: Write Back};

% V-pipe (secondary)
\node[pipeline] (pf_v) at (7, 20) {V-Pipe\\(Secondary)};
\node[pipeline] (d1_v) at (7, 19.2) {D1: Decode 1};
\node[pipeline] (d2_v) at (7, 18.4) {D2: Decode 2};
\node[pipeline] (ex_v) at (7, 17.6) {EX: Execute};
\node[pipeline] (wb_v) at (7, 16.8) {WB: Write Back};

% Pipeline flow
\draw[widearrow] (pf_u) -- (d1_u);
\draw[widearrow] (d1_u) -- (d2_u);
\draw[widearrow] (d2_u) -- (ex_u);
\draw[widearrow] (ex_u) -- (wb_u);

\draw[widearrow] (pf_v) -- (d1_v);
\draw[widearrow] (d1_v) -- (d2_v);
\draw[widearrow] (d2_v) -- (ex_v);
\draw[widearrow] (ex_v) -- (wb_v);

% Pairing rules
\node[note, text width=8cm] at (10.5, 18.9) {
\textbf{Pairing Rules (U+V):}\\
• Both must be simple (no complex addressing)\\
• No dependencies (RAW, WAR, WAW)\\
• V-pipe: Integer ALU only (no FPU, no shifts)\\
• No control flow in V-pipe\\
• Max 1 memory access per pair\\
• U-pipe: Any instruction\\
• V-pipe: Limited to: MOV, ADD, SUB, CMP, AND, OR, XOR, INC, DEC, LEA
};

\node[note, text width=8cm] at (10.5, 17) {
\textbf{Pairing Examples:}\\
✓ \texttt{ADD EAX, EBX} (U) + \texttt{MOV ECX, EDX} (V)\\
✓ \texttt{CMP EAX, 10} (U) + \texttt{INC EBX} (V)\\
✗ \texttt{ADD EAX, EBX} (U) + \texttt{SHL ECX, 4} (V) - shift not allowed\\
✗ \texttt{ADD EAX, [EBX]} (U) + \texttt{MOV ECX, [EDX]} (V) - 2 mem ops\\
✗ \texttt{JMP label} (U) + \texttt{ADD EAX, EBX} (V) - control flow
};

% ============================================================================
% LEFT COLUMN: Execution Units
% ============================================================================

\node[label] at (3, 15.5) {\textbf{Execution Units}};

% Integer ALU
\node[unit] (alu) at (3, 14.7) {Integer ALU\\(U-pipe)};
\node[note, text width=5cm] at (6, 14.7) {
• ADD, SUB, AND, OR, XOR\\
• INC, DEC, NEG, NOT\\
• CMP, TEST\\
• 1 cycle latency
};

% Integer ALU V-pipe
\node[unit] (alu_v) at (3, 13.7) {Integer ALU\\(V-pipe)};
\node[note, text width=5cm] at (6, 13.7) {
• Subset of U-pipe ops\\
• Simple ops only\\
• No shifts, no multiply
};

% FPU
\node[unit] (fpu) at (3, 12.7) {Floating Point Unit\\(U-pipe only)};
\node[note, text width=5cm] at (6, 12.7) {
• FADD: 3 cycles\\
• FMUL: 3 cycles\\
• FDIV: 39 cycles (FDIV bug!)\\
• Pipelined (3-stage)
};

% Load/Store
\node[unit] (mem) at (3, 11.5) {Load/Store Unit};
\node[note, text width=5cm] at (6, 11.5) {
• 64-bit data bus\\
• Aligned access: 1 cycle\\
• Misaligned: 2 cycles\\
• Max 1 memory op per clock
};

% Branch Unit
\node[unit] (branch) at (3, 10.5) {Branch Unit};
\node[note, text width=5cm] at (6, 10.5) {
• Dynamic branch prediction\\
• 256-entry BTB (Branch Target Buffer)\\
• 2-bit saturating counter\\
• Misprediction penalty: 3-4 cycles
};

% ============================================================================
% CENTER: Cache Hierarchy
% ============================================================================

\node[label] at (10, 15.5) {\textbf{Cache Hierarchy}};

% L1 Instruction Cache
\node[cache] (icache) at (10, 14.7) {L1 I-Cache\\8 KB, 2-way};
\node[note, text width=4cm] at (13, 14.7) {
• 32-byte lines\\
• Write-through\\
• 2-way set-associative
};

% L1 Data Cache
\node[cache] (dcache) at (10, 13.7) {L1 D-Cache\\8 KB, 2-way};
\node[note, text width=4cm] at (13, 13.7) {
• 32-byte lines\\
• Write-through\\
• 2-way set-associative
};

% Cache controller
\node[unit] (cache_ctrl) at (10, 12.7) {Cache Controller};
\node[note, text width=4cm] at (13, 12.7) {
• MESI protocol\\
• Snooping for SMP\\
• Write combining
};

% TLB
\node[cache] (itlb) at (10, 11.7) {I-TLB\\32 entries};
\node[cache] (dtlb) at (10, 10.9) {D-TLB\\64 entries};
\node[note, text width=4cm] at (13, 11.3) {
• 4 KB pages\\
• Fully associative\\
• TLB miss: page walk
};

% Bus Interface Unit
\node[unit] (biu) at (10, 10) {Bus Interface Unit\\64-bit data bus};

\draw[arrow] (icache) -- (cache_ctrl);
\draw[arrow] (dcache) -- (cache_ctrl);
\draw[arrow] (cache_ctrl) -- (biu);
\draw[arrow] (itlb) -- (biu);
\draw[arrow] (dtlb) -- (biu);

% ============================================================================
% RIGHT COLUMN: Register File and New Features
% ============================================================================

\node[label] at (17, 15.5) {\textbf{Register File}};

% General Purpose Registers
\node[regbox] (eax) at (17, 14.9) {EAX};
\node[regbox] (ebx) at (17, 14.4) {EBX};
\node[regbox] (ecx) at (17, 13.9) {ECX};
\node[regbox] (edx) at (17, 13.4) {EDX};
\node[regbox] (esi) at (17, 12.9) {ESI};
\node[regbox] (edi) at (17, 12.4) {EDI};
\node[regbox] (esp) at (17, 11.9) {ESP};
\node[regbox] (ebp) at (17, 11.4) {EBP};

% Segment Registers
\node[label, font=\footnotesize] at (17, 10.8) {Segment Regs};
\node[regbox] (cs) at (17, 10.3) {CS};
\node[regbox] (ds) at (17, 9.8) {DS};
\node[regbox] (ss) at (17, 9.3) {SS};

% Control Registers
\node[label, font=\footnotesize] at (17, 8.7) {Control Regs};
\node[regbox] (cr0) at (17, 8.2) {CR0};
\node[regbox] (cr3) at (17, 7.7) {CR3};

% Timestamp Counter (NEW!)
\node[label, font=\footnotesize] at (17, 7.1) {New in Pentium};
\node[regbox, fill=accentorange!20] (tsc) at (17, 6.6) {TSC (64-bit)};
\node[note, text width=3.5cm] at (17, 6) {
\textbf{RDTSC} instruction\\
Reads cycle counter\\
High-res timing\\
Ring 3 accessible
};

% ============================================================================
% BOTTOM LEFT: RDTSC Example
% ============================================================================

\node[label] at (3, 5) {\textbf{RDTSC Instruction Usage}};

\node[instr, minimum width=6cm, text width=5.5cm, align=left] at (3, 3.5) {
\textbf{Assembly:}\\
\texttt{rdtsc}  ; Read timestamp counter\\
\texttt{; EDX:EAX = 64-bit cycle count}\\
\\
\textbf{C Usage:}\\
\texttt{static inline uint64\_t rdtsc(void) \{}\\
\texttt{\ \ uint32\_t lo, hi;}\\
\texttt{\ \ asm volatile("rdtsc":"=a"(lo),"=d"(hi));}\\
\texttt{\ \ return ((uint64\_t)hi << 32) | lo;}\\
\texttt{\}}\\
\\
\textbf{Timing Example:}\\
\texttt{uint64\_t start = rdtsc();}\\
\texttt{/* code to measure */}\\
\texttt{uint64\_t end = rdtsc();}\\
\texttt{printf("Cycles: \%llu", end - start);}
};

% ============================================================================
% BOTTOM CENTER: APIC
% ============================================================================

\node[label] at (10, 5) {\textbf{APIC (Advanced Programmable Interrupt Controller)}};

\node[unit, minimum width=6cm] (apic) at (10, 4.2) {Local APIC};

\node[note, text width=5.5cm] at (10, 3.3) {
\textbf{Features:}\\
• Per-CPU interrupt handling\\
• Inter-Processor Interrupts (IPI)\\
• Local timer interrupt\\
• Performance monitoring counter\\
• Thermal sensor interrupt\\
\\
\textbf{Registers (MMIO):}\\
• Mapped at 0xFEE00000\\
• 256-byte register space\\
• Priority-based interrupt delivery
};

\node[unit, minimum width=6cm] (ioapic) at (10, 2.1) {I/O APIC (External)};

\node[note, text width=5.5cm] at (10, 1.3) {
\textbf{I/O APIC:}\\
• External chip (on motherboard)\\
• Routes IRQs to Local APICs\\
• Supports up to 24 IRQs\\
• Programmable routing\\
• Edge or level-triggered
};

% ============================================================================
% BOTTOM RIGHT: MMX Instructions (Pentium MMX variant)
% ============================================================================

\node[label] at (17, 5) {\textbf{MMX Instructions (P5 MMX)}};

\node[note, text width=6cm] at (17, 3.8) {
\textbf{MMX Registers:} (added in Pentium MMX, 1997)\\
• MM0-MM7 (64-bit each)\\
• Aliased to FPU ST(0)-ST(7)\\
• SIMD operations (Single Instruction, Multiple Data)\\
\\
\textbf{Data Types:}\\
• 8x 8-bit integers (packed bytes)\\
• 4x 16-bit integers (packed words)\\
• 2x 32-bit integers (packed dwords)\\
• 1x 64-bit integer (quadword)\\
\\
\textbf{Instructions:}\\
• PADDB/W/D: Packed add\\
• PSUBB/W/D: Packed subtract\\
• PMULL/H: Packed multiply\\
• PAND/OR/XOR: Packed logical\\
• PSLL/RL/RA: Packed shift\\
\\
\textbf{Use Cases:}\\
• Graphics (pixel operations)\\
• Video encoding/decoding\\
• Audio processing\\
• Image filtering
};

% ============================================================================
% PENTIUM FDIV BUG (Famous!)
% ============================================================================

\node[label] at (10, 0) {\textbf{Pentium FDIV Bug (1994)}};

\node[note, text width=18cm, fill=warningred!10, draw=warningred, thick] at (10, -1.5) {
\textbf{Description:} Floating-point division bug in early Pentium (60-100 MHz) caused incorrect results for certain inputs.\\
\\
\textbf{Example:} \texttt{4195835.0 / 3145727.0} should give \texttt{1.333820449136241002} but Pentium returned \texttt{1.333739068902037589}\\
\\
\textbf{Root Cause:} Lookup table in FPU had 5 missing entries (out of 1066). Used in SRT division algorithm.\\
\\
\textbf{Impact:} Intel initially downplayed issue ("affects 1 in 9 billion divisions"). Public outcry forced full recall.\\
\\
\textbf{Cost:} \$475 million USD replacement program. Major PR disaster for Intel.\\
\\
\textbf{Detection:} \texttt{cpuid} instruction returns family=5, model=1 or 2. Check for errata via BIOS update or OS flag.\\
\\
\textbf{Workaround:} Linux kernel detects and disables FPU if bug present, uses software floating-point emulation.\\
\\
\textbf{Fixed:} Pentium 75 MHz and later (stepping C1 and beyond) have corrected lookup table.
};

% ============================================================================
% PERFORMANCE CHARACTERISTICS
% ============================================================================

\node[label] at (3, -3.5) {\textbf{Performance Characteristics}};

\node[note, text width=8cm] at (3, -5) {
\textbf{IPC (Instructions Per Cycle):}\\
• Single-issue code: 0.5-0.8 IPC\\
• Dual-issue optimized: 1.5-1.8 IPC\\
• Ideal (all pairs): 2.0 IPC\\
\\
\textbf{Latencies:}\\
• Integer ADD/SUB: 1 cycle\\
• Integer MUL: 10 cycles (non-pipelined)\\
• FP ADD: 3 cycles (pipelined)\\
• FP MUL: 3 cycles (pipelined)\\
• FP DIV: 39 cycles (non-pipelined)\\
• L1 cache hit: 1 cycle\\
• L1 cache miss (to memory): 30-50 cycles\\
\\
\textbf{Branch Prediction:}\\
• Prediction rate: 80-90\% (depends on code)\\
• Misprediction penalty: 3-4 cycles\\
• No speculative execution (unlike P6)
};

\node[note, text width=8cm] at (10, -5) {
\textbf{Pentium vs 486:}\\
• 2x performance on integer code (dual-issue)\\
• 3-5x performance on FP code (pipelined FPU)\\
• Better branch prediction\\
• Larger caches (8KB vs 8KB unified on 486)\\
• Faster bus (60-66 MHz vs 33 MHz)\\
\\
\textbf{Pentium Variants:}\\
• Pentium 60/66: 0.8 \textmu m, 5V, 3.1M transistors\\
• Pentium 75-200: 0.6-0.35 \textmu m, 3.3V, split voltage\\
• Pentium MMX: 0.35 \textmu m, 2.8V, 4.5M transistors\\
• Pentium OverDrive: Socket upgrade for 486
};

\node[note, text width=8cm] at (17, -5) {
\textbf{System Integration:}\\
• Socket 4 (Pentium 60/66, 273-pin PGA)\\
• Socket 5 (Pentium 75-200, 320-pin SPGA)\\
• Socket 7 (Pentium MMX, 321-pin SPGA)\\
• Chipsets: 430FX (Triton), 430HX, 430VX\\
\\
\textbf{Operating Systems:}\\
• DOS (16-bit real mode)\\
• Windows 3.x (16-bit protected mode)\\
• Windows 95/98/NT (32-bit)\\
• Linux 1.x+ (full 32-bit support)\\
• MINIX 3 (uses protected mode, paging)\\
\\
\textbf{Compilers:}\\
• GCC: -march=pentium\\
• MSVC: /G5\\
• ICC: -mtune=pentium
};

% Background zones
\begin{pgfonlayer}{background}
    % Pipeline zone
    \node[fill=warningred!5, rounded corners, fit=(pf_u) (wb_v)] {};

    % Execution units
    \node[fill=accentorange!5, rounded corners, fit=(alu) (branch)] {};

    % Cache hierarchy
    \node[fill=secondarygreen!5, rounded corners, fit=(icache) (biu)] {};

    % Registers
    \node[fill=lightgray, rounded corners, fit=(eax) (tsc)] {};
\end{pgfonlayer}

% Annotations
\draw[decorate, decoration={brace, amplitude=8pt}]
    (0.5, 20) -- (0.5, 16.8) node[midway, left=10pt, align=center, font=\footnotesize] {5-stage\\dual-issue\\pipeline};

\draw[decorate, decoration={brace, amplitude=8pt, mirror}]
    (19, 14.9) -- (19, 11.4) node[midway, right=10pt, align=center, font=\footnotesize] {8 GPRs\\32-bit};

\end{tikzpicture}
\end{document}
